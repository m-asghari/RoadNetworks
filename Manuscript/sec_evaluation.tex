\section{Evaluating Probabilistic Predictions}
\vspace{0.2cm}
\label{sec:evaluate}
Evaluating the accuracy of probabilistic results is a task that has been either
over simplified or brushed under the carpet by many studies, focusing on the
technical aspects of managing uncertain data. However, we argue that this part
is crucial in order to justify the use of the uncertain aspects in the data,
rather than performing the traditional approaches for data cleaning (e.g., using the expected prediction
time). The evaluation of probabilistic prediction models for nominal values
(e.g., travel time) is a non-trivial problem that will be reviewed in this
section.

The problem statement is as follows: Let A be a prediction model that is able
to provide a probabilistic prediction $f_A^t$ of a value $v^t$ for a
time $t$ in the future and $f_A^t$ be a \textit{pdf} assigning a probability to
each possible value of $v^t$. Furthermore, let B be a prediction model which predicts
$f_B^t$ for the same value $v^t$. Our goal is to determine which of the two
models yields a better probabilistic prediction.

Two statistical methods are considered to be relevant to this problem:
goodness-of-fit tests \cite{Tay97} and scoring rules \cite{Bic07}. The
goodness-of-fit test describes how well a statistical model fits a set of
observations.
Measurements of goodness-of-fit typically summarizes the discrepancy between
observed values and the values expected under the model in question. An example is the
Pearson's Chi-Square test \cite{Pea00}, which can be used as a goodness-of-fit
test for a given \textit{pdf} $f$, which represents the theoretical behavior of
a random variable $v$, and an observed frequency distribution (sampled values)
from that variable. The test returns the probability of observing this frequency
distribution (or a frequency distribution with a higher difference) under the
assumption that the true distribution of $v$ follows $f$. Obviously this measure
might be used in order to compare two models $f_A$ and $f_B$. The main
difference of this setting is that the prediction is in our case time-dependent
and changes additionally depending on the current situation of the traffic on
an edge. Thus we are not able to draw a sufficiently large number of samples
that may validate either one or the other model.
Scoring models on the other hand are designed to evaluate probabilistic
prediction models that are time dependent. Fields of application are weather
forecasting or betting games. The most prominent representative is known as the
Brier scoring rule \cite{Bri50}, which is described as follows. Let $f^t$ be a
probabilistic prediction model (e.g., a probability mass function) for a
variable, whose true value $v^t$ is one of several classes $C$ (e.g., ``tomorrow
is sunny (75\%) or rainy (25\%)), then for one specific $t$ the Brier scoring
rule returns:

$$
	S^{Brier}(f^t, v^t) = -\sum_{c \in C} (f^t(c) - I(v^t = c))^2
$$
where $f^t(c)$ represents the probability that is assigned to class $c$ and
$I(v^t = c)$ returns 1 if the true value of $v^t$ is equals $c$.
Intuitively, this scoring function gives a reward dependent on the
probability that the probabilistic prediction model assigns to the true value.
Summing up these rewards over several times $t \in T$ yields a score that is
higher the better the prediction model $f^t$ predicts the true probability
distribution of $v^t$ at each value of $t$:

\vspace{0.1cm}

$$
	S^\Sigma (f, v) = \sum_{t \in T} S^*(f^t, v^t)
$$

\vspace{0.1cm}

With this mechanism, it is possible to evaluate two probabilistic prediction
models $f_A$ and $f_B$ by comparing their corresponding rewards.
Scoring rules can also be used in our scenario, however they are originally
designed for categorical values (e.g., sunny, cloudy, rainy) rather than ordinal
values (e.g. travel time). Thus these scores are not sensitive to distance,
meaning that no credit is given for assigning high probabilities to values near
but not identical to the one materializing (e.g., the true outcome). For example
a probabilistic prediction model A, that predicts 5 minutes (90\%) or 6
minutes (10\%) and a model B that predicts 6 minutes (10\%) or 10 minutes
(90\%), both obtain the same score when the true travel time is 6 minutes. However, we argue that this approach
does not account for the ordinal nature of travel times since a true travel time
of 6 minutes is rather close to the highly probable 5 minutes of model A and a
good scoring rules should thus favor model A over B. For this reason we propose
to use the continuous ranked probability score (CRPS) which is defined as
follows \cite{Her00}:

\vspace{0.1cm}

\begin{equation}
CRPS(f^t, v^t) = \int_{-\infty}^{+\infty} \int_{-\infty}^{x} f^t(y) dy - I(x
\geq v^t) dx
\end{equation}

\vspace{0.1cm}

CRPS thus expresses some kind of distance between the probabilistic forecast
$f^t$ and truth $v^t$. Another very useful property of CRPS is that it can be computed in
linear time (in the number of possible outcomes) in the case when $f^t$ is given
by a \textit{pmf} and has a closed form for the case where $f^t$ is given by a
normal distribution $\mathcal{N}$ with parameters $\mu$ and $\sigma$ \cite{GneRaf07}:

\vspace{0.1cm}

\begin{multline}
CRPS(\mathcal{N}(\mu,\sigma), v^t) = \sigma \left[ 2\varphi\left(\frac{x -
\mu}{\sigma}\right) - \frac{1}{\sqrt{\pi}} + \right. \\ \left.\frac{x -
\mu}{\sigma}\left(2 \Phi\left(\frac{x - \mu}{\sigma}\right) - 1 \right) \right]
\end{multline}

\vspace{0.2cm}

where $\varphi$ and $\Phi$ denote the probability density function and the
cumulative distribution function of a standard Gaussian variable. One advantage
of the CRPS is that it reduces to the mean absolute error (MAE) if the forecast
is deterministic. In practice, this makes it possible to compare an
ensemble forecast with a deterministic forecast of the same variable in a
consistent fashion. Since for the experimental evaluation we discretize the time
to 1 second (i.e., $\phi$ = 1 sec) this means that the CRPS score can be
interpreted as the mean absolute error of the predicted travel time in seconds.

