\section{Conclusions and Future Work}
\label{sec:conclusion}
In this work we studied the problem of assigning reliability or confidence values to the estimated travel time of a path in road networks. We showed that in order to achieve this goal, a whole pipeline of methods has to be present. We developed three novel methods to obtain probabilistic link travel times for each link in the network, using historical and current traffic observations. We presented several methods for summing up the travel times of each link on a given path under different models. We also introduced a method for evaluating probabilistic link and path travel time estimations. This technique made it for the first time possible to compare the existing approaches and the developed link travel time estimation methods against each other in terms of accuracy.

The results presented in this paper leave room for improvement and further investigation. In the current study, we only consider traffic data from the L.A. metropolitan area. Some of the parameters we compute in our experiments (e.g., the road network forgets the traffic from 1 hour ago) may end up having different values on road networks in other parts of the globe. It would be interesting to investigate whether such parameters have the same values on other road networks as well and if not what may cause the difference. In addition, a follow up study on the efficiency of calculating/storing/evaluating pltts (either via pre-processing or real-time updates) and directly applying them within industrial-grade route-planning algorithms would be of significant practical value in this domain.
