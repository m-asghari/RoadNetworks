\section{Conclusions and Future Work}
\label{sec:conclusion}
In this work we studied the problem of assigning reliability or confidence
values to the estimated travel time of a path in road networks. We showed
that in order to achieve this goal, a whole pipeline of methods has to be
present. We developed three novel methods to obtain probabilistic link
travel times for each link in the network, using historical and
current traffic observations. We presented several methods for summing up
the travel times of each link on a given path under different models. We also
introduced a method for evaluating probabilistic link and path travel time
estimations. This technique made it for the first time possible to compare the
existing approaches and the developed link travel time estimation methods
against each other in terms of accuracy.

The results presented in this paper leave room for improvement and further
investigation. In the current methods, we treated all weekdays as equivalent,
but our observations showed that there is a difference that should be
analyzed and taken into account (e.g. traffic is usually worse on monday
mornings than on other days due to out-of-town commuters). Another important
point for further investigation is the notion of correlation. In following
studies we want to analyse the influence of time-dependent correlations and the
introduction of more possible states of a link. Last, the focus of this work was
mainly on the accuracy comparison of different approaches and thus we assumed
all necessary data to be present in main-memory. However, in a larger setting
this may not be possible and the prediction has to run efficient for the data
to be present in secondary memory database. In this case, we envision
summarization data structures that allow for efficient prediction, based on the
current situation.
