\section{Reliability Computation}
\label{sec:methods}
In this section we will describe several approaches to compute the reliability
of a given path in a network $G(V, E, S)$. We will start by discussing the
simple case of a static network with uncertain independent link travel times and
successively increases the complexity of the underlying graph by adding the
attributes of time-dependency and correlation between link travel times.

\subsection{Probabilistic Link Travel Times}
\label{subsec:static}
The most important design decision for supporting the computation of reliable
paths is how to model the uncertainty of link travel times. 

\textbf{Continuous Representation: }Under the assumption that link travel times
are normally distributed and independent, the path travel time is also a random variable which is normally
distributed with a mean and a variance, given by

\begin{gather}
\label{eq:normal}
	\mu_{p} = \sum_{(i,j)\in p} \mu_{ij} \text{ and } \sigma_{p}^2 =
	\sum_{(i,j)\in p} \sigma_{ij}^2 
\end{gather}

where $(i,j)$ represent links on path $p$. The computational complexity is thus
$O(N)$ where $N$ is the number of links on path $p$.

% \subsection{Discrete Representation: }
% \label{subsec:discrete}
% Another approach is to represent the link travel times by discrete pmfs 
% (probability mass functions). In this case the time domain has to be
% discretized. The simplest discretization scheme, known as b-discrete divides the
% time domain $T = \{t | t = n\cdot \phi \wedge n \in \NN \}$ evenly into
% intervals of length $\phi$.
% The corresponding probability mass function $F_{ij}$ of link travel time
% $c_{ij}$ reads
% 
% \begin{equation}
% 	F_{ij}(b) = \begin{cases}\int_b^{b+\phi}f_{ij}(w)dw \qquad b =
% 	0,\phi,\ldots, (L-1)\phi\\
% 	\int_b^{\infty}f_{ij}(w)dw \qquad b =
% 	L \phi\\
% 	0 \qquad otherwise
% 	\end{cases} 
% \end{equation}
% 
% where $L \phi$ is the maximal considered time horizon in the future. 

\textbf{Discrete Representation: } Given the \textit{pmf} $F_{ij}$ of the link travel
time for the link $(i,j)$ we can compute the \textit{pmf} $J_{sj}$ of the path travel time of path $p_{si}$ extended by the link
$(i,j)$ by using the poisson-multinomial recurrence.

\begin{equation}
\label{eq:pmr}
	J_{sj}(b) = \sum_{h=0}^b J_{si}(b-h) F_{ij}(h)  , \forall b = 0, \phi,\ldots, L
	\phi
\end{equation}

This equation can be computed efficiently by starting with the first link
$(s,i)$ of the path $p_{sd}$ and its corresponding \textit{pmf} $F_{si}$. The adjacent
link $(i,j)$ is considered next. The \textit{pmf} $J_{sj}$ is computed according to
Equation \ref{eq:pmr} and the process repeats with the next adjacent link until
node $d$ is reached. When implementing this, the algorithm obviously does not
always have to consider each $b$ of the time horizon, but only all $b$'s corresponding to
the minimum and the maximum possible time to traverse the path $p_{si}$ and link
$(i,j)$. For each link on the path we have to perform $O(L^2)$
computations, thus the overall complexity to compute $J_{sd}$ is $O(N \cdot
L^2)$ where $N$ is the number of links on the path. 

\subsection{Time-Dependency}
\label{subsec:time}
So far we assumed that the travel time $c_{ij}$ for a link is static, however,
in a real world traffic network the link travel times differ drastically
depending on time of the day and day of the week. Failing to integrate this
knowledge yields a high degree of variance in the possible outcomes and thus an
imprecise result which might not be useful for a user starting at a specific
time. Thus we integrate this concept to stochastic time-dependent networks,
where the link travel time $c_{ij}^t$ depends on the arrival time $t$ at node
$i$. We discuss both, discrete and continuous modelling under time-dependency in the following.

\textbf{Discrete Representation: } The work
described in \cite{Nie09b} is closest to this setting, however, it
focuses on a different problem setting and only briefly considers time-dependent
link travel times. Thus the results can not be straightforwardly used.
Therefore, some reformulations have to be made to fit with our problem setting.
We generalize the technique described in [25]  to retrofit to our problem as follows.  
Let $F_{ij}^t$ be the \textit{pmf} describing the link travel time of the link $(i,j)$ at
time $t$ and $J_{si}$ be the path travel time for a path $p^{si}$ w.l.o.g. we
assume that the route starts at $s$ at $t = t_c = 0$ then

\begin{equation}
\label{eq:pmr2}
	J_{sj}(b) = \sum_{h=0}^b J_{si}(b-h) F_{ij}^{b-h}(h)  , \forall b = 0,
	\phi,\ldots, L
	\phi
\end{equation}

Similar to Equation \ref{eq:pmr} we can build an incremental algorithm upon this
equation until node $d$ is reached. The computational complexity stays the same
than in the time-independent case $O(N \cdot L^2)$. An example for the
computation for two consecutive street segments is given in the following:

\ifthenelse{\boolean{TR}}{
\begin{example}
Consider the uncertain time-dependent link travel times for two edges $(s,i)$
and $(i,j)$ in Table \ref{tab:example}. Link travel times for times
which are not needed for the computation are omitted. The \textit{pmf} of the path
travel time $\pi_{sj}^0$ is given by: 
\begin{equation*}
\begin{split}
J_{sj}(3) &= F_{si}(3-2) \cdot F_{ij}^{3-2}(2)\\
&= 0.2 \cdot 0.9 = 0.18\\
J_{sj}(4) &= F_{si}(4-2) \cdot F_{ij}^{4-2}(2) + F_{si}(4-3) \cdot
F_{ij}^{4-3}(3)\\ 
&= 0.5 \cdot 0.8 + 0.2 \cdot 0.6  = 0.52\\
J_{sj}(5) &= F_{si}(5-2) \cdot F_{ij}^{5-2}(2) + F_{si}(5-3) \cdot
F_{ij}^{5-3}(3)\\ 
&= 0.3 \cdot 0.2 + 0.5 \cdot 0.2  = 0.16\\
J_{sj}(6) &= F_{si}(6-3) \cdot F_{ij}^{6-3}(3)\\
&= 0.3 \cdot 0.7  = 0.21\\
J_{sj}(7) &= F_{si}(7-4) \cdot F_{ij}^{7-4}(4)\\ 
&= 0.3 \cdot 0.1  = 0.03
\end{split}
\end{equation*}
\end{example}

\begin{table}
\begin{center}
  \begin{tabular}{| l || c | c |}
    \hline
   t & $c_{si}^t$ & $c_{ij}^t$\\
    \hline    \hline
0 & 1 (0.2), 2 (0.5), 3 (0.3) & ---\\ \hline
1 & --- & 2 (0.9), 3 (0.6)\\ \hline
2 & --- & 2 (0.8), 3 (0.2)\\ \hline
3 & --- & 2 (0.2), 3 (0.7), 4 (0.1)\\ \hline

  \end{tabular}
\end{center}
\caption{Example pmfs of link travel times}
\label{tab:example}
\vspace{-0.5cm}
\end{table}
}{}

\textbf{Continuous Representation: } For link travel times represented by
time-dependent continuous distributions the closest approach is
\cite{Bi-Yu13}. However, as the authors pointed out, there is no closed
form for the exact continuous distribution of the path travel time.


\subsection{Correlated Link Travel Time}
\label{subsec:cor}
% For the continuous case under the assumption of normal distributed link travel
% times \cite{SesSri10} provides a general solution. In this model any pair
% $((i,j), (k,l))$ of edges in the network may be correlated and the corresponding
% correlation $corr(c_{ij},c_{kl}) = \rho_{ij-kl} \in \boldsymbol\Sigma$ is stored
% in a covariance matrix. 
As discussed earlier, link travel times of edges are certainly
correlated. Modelling this correlation is thus seen to be crucial in order to
obtain meaningful results. As mentioned, this work proposed a discretization
approach as a solution to this problem.
Since this is equivalent to the above discrete case we will not discuss it
separately.

\textbf{Continuous Representation: } We manage the continuous representation by
extending Eq. \ref{eq:normal} as follows:


\begin{gather}
\label{eq:normal2}
	\mu_{p} = \sum_{(i,j)\in p} \mu_{ij}, \\
	\sigma_{p}^2 = \sum_{(i,j)\in p} \sigma_{ij}^2 +\sum_{(i,j)\neq(k,l)\in p}
	\rho_{ij-kl}
	\sigma_{ij} \sigma_{kl}
\end{gather}

Obviously this compact representation of the correlation of link travel times is
only possible due to the assumption of normally distributed link travel times.
Under this model, however, the complexity of the computation of the reliability
of a path is $O(N^2)$.

\textbf{Discrete Representation: } Following \cite{Nie06}, we compute the
path travel time accordingly.

\begin{multline}
\label{eq:pmr3}
	J_{sj}(b,z) = \sum_{h=0}^b \sum_{r=0}^1 \alpha^{rz}_{ij} J_{si}(b-h,r)
	F_{ij}(h,z)  , \\ \forall b = 0, \phi,\ldots, L
	\phi, z = \{0, 1\}
\end{multline}

where $F_{ij}(h,z)$ is the \textit{pmf} representing the travel time $c_{ij}$ under
normal ($z=0$) or congested ($z=1$) scenario. Including this correlation
parameter increases the runtime complexity to $O((N \cdot L^2 \cdot |S|^2)$
where $|S|$ is the number of considered states. However \cite{Nie06} and
other works argue that $|S|$ should be set to 2.

\subsection{Time-Dependency and Correlations}
\label{subsec:timcor}
\vspace{0.2cm}

\textbf{Continuous Representation: } To the best of our knowledge there does not
exist any work that incorporates both, time dependency and correlation for
continuous representation of link travel times.

\textbf{Discrete Representation: } However, \cite{Nie09a} offers a solution
for the discrete representation  where the authors combine the ideas from \cite{Nie09b} and
\cite{Nie06}:
\begin{multline}
\label{eq:pmr4}
	J_{sj}(b,z) = \sum_{h=0}^b \sum_{r=0}^1 \alpha^{rz}_{ij} J_{si}(b-h,r)
	F_{ij}^{b-h}(h,z)  , \\ \forall b = 0, \phi,\ldots, L
	\phi, z = \{0, 1\}
\end{multline}

The runtime complexity of this approach is equivalent to the time-independent
case $O((N \cdot L^2 \cdot |S|^2)$.
