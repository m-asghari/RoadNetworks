\section{Route Reliability Computation}
\label{sec:methods}
In this section we present a classification of existing approaches for computing route pdfs. Towards this end, we identified three basic characteristics of existing algorithms which become the basis of our classification; link representation, time dependency and link correlation. Different algorithms model link travel times as either discrete or continuous distributions, they can consider the network time dependent or static and finally they would either consider correlation between the travel time of different links or look at them independently. \cref{tab:methods} summarizes the existing work on route pdfs (\textbackslash pmfs) based on the three attributes we presented here. A brief description of each work is presented in \cref{sec:related}.

\begin{table}
\centering
\begin{tabular}{| l || l | c | c | c | c|}
\hline
Ref. & Method & Link model & Time-dep. & Correl. \\
\hline    \hline
\cite{Frank69} & ConStaInd & continuous  & no & no\\ \hline
               & DisStaInd & discrete & no & no\\ \hline
               & ConTimInd & continuous & yes & no\\ \hline
\cite{Miller-Hooks98,Miller-Hooks00, Nie09b} & DisTimInd & discrete & yes & no\\ \hline
\cite{Seshadri10, Zockaei13,Bi-Yu13} & ConStaCor & continuous & no & yes\\ \hline
\cite{Nie06,Hua10,Fan05} & DisStaCor & discrete & no & yes\\ \hline
\cite{Dong12} & ConTimCor & continuous & yes & yes\\ \hline
\cite{Nie09a} & DisTimCor & discrete & yes & yes\\ \hline
\end{tabular}
\caption{Models considered in this work}
\label{tab:methods}
\end{table}

To fully evaluate the techniques in \cref{sec:lttestimation} for computing \textit{pltts}, we need to apply each technique to existing algorithms and check the accuracy of the resulting route pdf (\textbackslash pmf). Also, since there has not been any study to compute \textit{pltts}, it has not been possible to compare the performance of these algorithms based on real world datasets. Following, we briefly explain a few classes of the algorithms presented in \cref{tab:methods} which we later use in our experiments. The selection of the classes is such that it will give us the ability to compare the effectiveness of the attributes used to classify these algorithms. For example by comparing the results of \textbf{ConStaInd} (\textbf{Con}tinuous, \textbf{Sta}tic, \textbf{Ind}ependent) with the results of \textbf{ConStaCor} we can find out whether considering link correlations increases the accuracy of the final pmf or not.

\textbf{ConStaInd: }Assuming link travel times are normally distributed and independent, the path travel time is also a random variable which is normally distributed with a mean and a variance, given by

\begin{gather}
\label{eq:normal}
	\mu_{p} = \sum_{(i,j)\in p} \mu_{ij} \text{ and } \sigma_{p}^2 =
	\sum_{(i,j)\in p} \sigma_{ij}^2 
\end{gather}

where $(i,j)$ represent links on path $p$.

\textbf{DisStaInd: } Given the \textit{pmf} $F_{ij}$ of the link travel time for the link $(i,j)$ we can compute the \textit{pmf} $J_{sj}$ of the path travel time of path $p_{si}$ extended by the link $(i,j)$ by using the poisson-multinomial recurrence.

\begin{equation}
\label{eq:pmr}
	J_{sj}(b) = \sum_{h=0}^b J_{si}(b-h) F_{ij}(h)  , \forall b = 0, \phi,\ldots, L
	\phi
\end{equation}

This equation can be computed efficiently by starting with the first link $(s,i)$ of the path $p_{sd}$ and its corresponding \textit{pmf} $F_{si}$. The adjacent link $(i,j)$ is considered next. The \textit{pmf} $J_{sj}$ is computed according to Equation \ref{eq:pmr} and the process repeats with the next adjacent link until node $d$ is reached. 

\textbf{DisTimInd: } Let $F_{ij}^t$ be the \textit{pmf} describing the link travel time of the link $(i,j)$ at time $t$ and $J_{si}$ be the path travel time for a path $p^{si}$. We assume that the route starts at $s$ at $t = t_c = 0$ then:

\begin{equation}
\label{eq:pmr2}
	J_{sj}(b) = \sum_{h=0}^b J_{si}(b-h) F_{ij}^{b-h}(h)  , \forall b = 0,
	\phi,\ldots, L
	\phi
\end{equation}

Similar to Equation \ref{eq:pmr} we can build an incremental algorithm upon this equation until node $d$ is reached.

\textbf{ConStaCor: } We manage to capture link correlations for the continuous representation by extending Eq. \ref{eq:normal} as follows:

\begin{gather}
\label{eq:normal2}
	\mu_{p} = \sum_{(i,j)\in p} \mu_{ij}, \\
	\sigma_{p}^2 = \sum_{(i,j)\in p} \sigma_{ij}^2 +\sum_{(i,j)\neq(k,l)\in p}
	\rho_{ij-kl}
	\sigma_{ij} \sigma_{kl}
\end{gather}

This compact representation of the correlation of link travel times is only possible due to the assumption of normally distributed link travel times. 