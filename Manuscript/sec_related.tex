\section{Related Work}
\label{sec:related}
In this section we will review related work on the problem of computing reliability of routes.

\subsection{Deterministic Link Travel Times}
The problem of point-to-point fastest path computation in static road networks
is extensively studied with many precomputation techniques proposed to speed-up
the computation (e.g., \cite{Samet08,Goldberg05}). These studies
make the simplifying assumption that travel-times of the network edges are single constant deterministic values. As we discussed, in real-world road networks the edge travel-times are time-dependent with a degree of uncertainty, where the arrival-time on an edge determines the
actual travel-time on it. Cooke and Halsey \cite{Cooke66} first studied time-dependent fastest
path probem (TDFP) considering the arrival times to edges. They solved the
problem using Dynamic Programming in discrete time. In \cite{Dreyfus69}, Dreyfus
showed that TDFP problem can be solved by a generalization of Dijkstra's method as efficiently as for static fastest path problems.
However, Halpern \cite{Halpern77} proved that the generalization of Dijkstra's algorithm is only true for FIFO networks. The FIFO property in road networks ensures that
vehicle-A starting at some node, always arrives at its destination before
vehicle-B, taking the same route at a later time.

In recent years, an increasing number of navigation companies (e.g., Navteq and
Inrix) have started releasing their time-dependent travel-time information for
road networks. With the availability of such datasets, the research community focused on
efficient computation of time-dependent fastest paths and introduced
precomputation techniques (e.g., \cite{Nannicini08,Delling08,
Batz10,Demiryurek11}). The main idea of these
techniques can be grouped in two categories. The first group 
(\cite{Nannicini08,Demiryurek11}) proposed  to solve
TDFP with bidirectional A* with different precomputed heuristic functions. On
the other hand, the second group
(\cite{Delling08,Batz10})  focused on a method that
removes unimportant nodes from the graph without changing the fastest path distances between the remaining (more important) nodes.

% While TDFP approaches yield more accurate paths than that of static algorithms,
% they fall short to capture the inherit uncertainty of the edge travel-times.
% This is because all of the existing time-dependent fastest path algorithms
% model the road network with deterministic time-varying edge
% weights. The travel-times are usually computed by averaging the
% historical travel-times to assign a deterministic cost (i.e., travel-time)
% for each time instance (e.g., for every 15 minutes). There also exists some
% studies that combine the historical and real-time traffic information
% (\cite{PanDS12}) to capture the uncertainty of travel-times and generate
% more up-to-date travel-times. However, the travel-times are still deterministic
% as they are assigned a single value based on the prediction of the short-term
% impact of the real-time traffic.

\subsection{Uncertain Link Travel Times}
\label{subsec:ult}
Uncertainty in databases has attracted a lot of attention from the research
community in the last decade . Many approaches for managing
\cite{Agrawal06,Antova08,Jampani08}, querying
\cite{Soliman07,Li11} and analysing \cite{Chui07,Chau06}
these uncertain datasets have been proposed.
Especially in the field of spatial \cite{Cheng04,Dai05} and
spatio-temporal \cite{Emrich12,Niedermayer13} data, uncertainty was considered
as an important issue due to uncertain sensor measurements or
probabilistic estimations of unknown values. Although these plethora of works
provide many efficient techniques to deal with the uncertainty, only a few works
actually consider how to obtain this uncertainty in a way such that the outcome
of these techniques is meaningful and evaluate the whole end-to-end pipeline of
this process.

Though there exists a large body of work on networks with certain link travel
times, these approaches are not adaptable when the travel times become
uncertain due to sensor failures or predictions of future states of the traffic
network.

The first work \cite{Frank69}  dealing with uncertainty in this context considers
a graph with uncertain edge weights with arbitrary distribution. The particular
interest of this work is to study the characteristics of the probability
distribution describing the length of the shortest path between a source and a
destination.
The author considers the problem from a statistical point of view and no algorithm (besides
a Monte-Carlo driven sampling approach) to compute this distribution is
provided.
The work concludes that (under certain assumptions) the length of the shortest
path is approximately normally distributed. In the end the interesting problem of
comparing two paths is presented. Since a comparison based on the expected
length might not be satisfying the author proposes to compare the paths based on
the probability not to be longer than a parameter $l$. An examplary query
corresponding to this setting would be: \textit{``Which of two possible paths
from my home to work is more likely to let me arrive before 10:00am''}. A
solution is presented under the assumption that the length of a path is normally
distributed.
In contrast to the above work, recent studies have considered more
sophisticated scenarios where the link travel times are not only uncertain but
also time varying and correlated.

\subsubsection{Time-Dependency}
Time dependency of link travel times was recently studied in
large scale traffic networks \cite{Demiryurek11,Pan12} without uncertainty.
It has been shown that considering the predicted value of a link travel time at the arrival at
that link rather than its value at start of the route vastly improves the accuracy of fastest route
algorithms. One work impressively demonstrating this, is \cite{Yuan13} where
the authors utilize historical taxi trajectories in order to learn the
time-dependent travel times of a road network and show the superiority to
routing services like Google or Bing Maps.

Miller-Hooks and Mahmassani first considered stochastic,
time-varying networks where link travel times are independent random variables
with time dependent (discrete) probability distributions \cite{Miller-Hooks98}.
They address the problem of finding the least possible time path in networks. Specifically, an
algorithm is proposed for determining the least possible travel time from every node to the destination node, its corresponding path and the corresponding probability of
occurrence of that travel time. A sample query would be: \textit{``Regardless
of my starting time, provide me the fastest possible (under best conditions)
route and its corresponding travel time probability from all nodes (intersections) to
work''}. The resulting least possible time path from any starting point may
however have a vanishingly small probability and thus not even be relevant to
the user.

The same authors later addressed the problem of least expected
time (LET) paths under the same settings in \cite{Miller-Hooks00}. The problem is to
determine the a priori least expected time paths with the associated expected time from all nodes to
a given destination for each departure time interval
in the period of interest. The sample query would thus slightly differ from the
previous one:\textit{``Provide me the fastest route from all nodes
(intersections) to work under average conditions when starting between 9:00am
and 10:00am''}. In time-invariant stochastic networks this
problem is trivial since it can be reduced to the static case with certain link
travel times by providing each edge with its expected travel time and solving
the shortest path problem using existing methods. In the time variant network
setting however this problem is more complex and two algorithms based on
label-correcting approaches are proposed. The first one provides an exact
solution yielding a worst-case non-polynomial but average polynomial complexity
whereas the second one is more efficient but provides only a lower bound of the
expected travel time and no path information for the LET paths.

In a different study, Nie and Wu \cite{Nie09b} proposed an algorithm for
solving the shortest path problem with on-time arrival reliability (SPOTAR). The SPOTAR problem is to
find the latest possible departure time and the associated route to attain a
given probability of arriving at the destination at a specified arrival time or
earlier. A query example would be: \textit{``What is the latest time I have to
leave home and what path do I have to take to be at work at 10:00am or earlier
with a probability of 90 \%''}. The authors model link travel times by independent
discrete probability distributions and propose an extension to the time-dependent case.
The problem is solved using an exact-label-correcting algorithm whose complexity
is known to be non-deterministic polynomial.

In \cite{Bi-Yu13} the authors assume normally distributed link
travel speeds. They note that some models assume that the travel time depends only
on the arrival time and do not consider the possibility that the travel time might change while the vehicle is on the link. Such models suffer from the drawback that they are not
statistical FIFO consistent. This problem is solved by relaxing this assumption and even allowing
linear correlations between link travel times at consecutive times. In this
case, however, there does not exist a closed form for the exact continuous
distribution of the link travel time, thus a discrete model is proposed. The
ultimate goal is to find the least expected time path (cf \cite{Miller-Hooks00})
but between a given source and destination rather than all possible sources. Due
to the problem setting, no probabilistic guarantees (reliability) can be assigned to the resulting path.


\subsubsection{Correlation}
Link travel times in traffic networks are generally dependent on each
other. For example a traffic jam usually extends over several road segments
and a high travel time on one segment usually implies a high travel time on the
next segment. The consideration of correlation between link travel times of
different edges is thus certainly relevant in traffic networks.

In \cite{Nie06} the authors propose an adaptive path finding strategy for the
stochastic on-time arrival (SOTA). The main idea of SOTA is to provide
the user the route with the highest probability to arrive at the
destination before a given time threshold. A sample query for SOTA is:
\textit{``Which route makes me arrive at work before 10:00am with the highest
probability?''}. They use a discrete model to represent link travel times and consider a simple form of
correlation between adjacent links in which the travel time of an edge is
modeled by two probability mass functions that represent the normal and the
congested state. The states of two adjacent links are then correlated by
conditional probabilities. An algorithm yielding polynomial runtime complexity
is proposed.
Based on the uncertainty model of \cite{Nie06}, the authors of
\cite{Hua10} proposed an approximation algorithm for the probabilistic path
travel time and heuristic approaches to find the best path(s). Though the study
evaluated the approximation error of the inexact approach on a large
real-world dataset they did not evaluate the validity of the exact outcome.

 A similar study \cite{Fan05} presents an
algorithm based on dynamic programming in order to return the path from a given source to a destination
with the least expected travel time (LET). The link travel times are given by
\textit{pmf}s and correlations are modeled similar to \cite{Nie06}.

Another work in this direction is \cite{Seshadri10}, where the authors consider
the problem of finding the optimal reliability path (ORP), which is equivalent
to the SOTA problem. In their setting the travel time of an edge is normally
distributed and travel times of different edges may correlate without changing
over time. Under these assumptions, subpath optimality (each subpath of the
result path is optimal regarding some criterion) of conventional approaches
does not hold, and hence has to be adapted. Since the proposed algorithm is
also based on a label correcting algorithm, the solution requires the
enumeration of all possible exponential paths in the worst-case. Due to this
characteristic, a fallback solution based on Monte-Carlo sampling is proposed to provide an
approximate solution.

\cite{Zockaei13} follows the same idea of sampling and provides an
approximate solution using a Monte-Carlo based approach to the SPOTAR (cf
\cite{Nie09b}) problem. Time-dependency is not considered. The main
statement of the simulation based study is that even for the static case, not
considering correlations between edge-weight yields considerably different and unpredictable
results in real-world problems.

\subsubsection{Time-Dependency and Correlation}
Including both, time-dependency and correlation of link travel times for route
optimization problems is generally a very complex undertaking and has only been
considered in very recent studies.

In \cite{Dong12}, time-dependent road networks with normally distributed
link travel times and local correlations between them based on two states
(similar to \cite{Nie06}) are considered. The goal here is to find the path
with the lowest percent variation (PV) between a source and a destination, where
PV is defined by the quotient of the standard deviation and the mean travel time of
the path. The proposed algorithm yields an approximation of the exact result,
but no runtime complexity or runtime experiments are provided.

In \cite{Nie09a}, the authors extend their previous work \cite{Nie09b} by
introducing limited spatial correlations into the SPOTAR problem - referred to
as reliable apriori shortest path problem (RASP). Specifically, the probability mass function of the traversal time on a link is assumed to be
conditional on the state of that link. The states of a link have a Markovian
property. Namely, the state of present link is dependent on the state of the
link traversed right before arriving at its starting node, and independent of
the links traversed prior to that. The probability distribution of link
traversal times is also allowed to vary over time, which provides a mechanism to
account for the dynamic network behavior such as congestion effects caused by
rush hour traffic. Since the problem is shown to be non-deterministic
polynomial, an approximation algorithm is proposed.

Table \ref{tab:methods} summarizes and categorizes all aforementioned approaches
for handling probabilistic link travel times.

\begin{table}
    \centering
  \begin{tabular}{| l || l | c | c | c | c|}
    \hline
    Ref. & Method & Link model & Time-dep. & Correl. \\
    \hline    \hline
\cite{Nie06} & SOTA & discrete  & no & yes\\ \hline
\cite{Seshadri10} & ORP & normal & no & yes\\ \hline
\cite{Nie09b} & SPOTAR & discrete & yes & no\\ \hline
\cite{Nie09a} & RASP & discrete & yes & yes\\ \hline
\cite{Bi-Yu13} & LET & normal & no & yes\\ \hline
\cite{Zockaei13} & SPOTAR & gamma & yes & yes\\ \hline
\cite{Dong12} & LET & normal & yes & yes\\ \hline
  \end{tabular}
  \caption{Models considered in this work}
  \label{tab:methods}
  \end{table}


\subsection{Prediction of Link Travel Times}
The prediction of link travel times is an essential part for the
real time computation of the path travel time in a time-dependent
traffic network. Thus, there exist a large body of work that solely
concentrates on techniques solving this problem based on
single link models \cite{Pan12}, traffic incident models
\cite{Pan13}) or Bayesian network models \cite{Sun06}. Neither
of these approaches does however incorporate the inherent uncertainty of the
predicted values and does not allow for the use of the algorithms discussed in
Section \ref{subsec:ult}. Recently Yang et al.\cite{Yang13} proposed to
utilize spatio temporal hidden markkov models for this purpose. However the
construction of the model is costly and the prediction
process does not include the current situation in the network which makes the approach hardly
applicable in the real-time setting considered in this work.
